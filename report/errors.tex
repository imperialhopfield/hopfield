
%%%%%%%%%% Start TeXmacs macros
\newcommand{\mathpi}{\pi}
\newcommand{\nin}{\not\in}
\newcommand{\nosymbol}{}
\newcommand{\tmem}[1]{{\em #1\/}}
\newcommand{\tmkbd}[1]{\texttt{#1}}
\newcommand{\tmfloatcontents}{}
\newlength{\tmfloatwidth}
\newcommand{\tmfloat}[5]{
  \renewcommand{\tmfloatcontents}{#4}
  \setlength{\tmfloatwidth}{\widthof{\tmfloatcontents}+1in}
  \ifthenelse{\equal{#2}{small}}
    {\ifthenelse{\lengthtest{\tmfloatwidth > \linewidth}}
      {\setlength{\tmfloatwidth}{\linewidth}}{}}
    {\setlength{\tmfloatwidth}{\linewidth}}
  \begin{minipage}[#1]{\tmfloatwidth}
    \begin{center}
      \tmfloatcontents
      \captionof{#3}{#5}
    \end{center}
  \end{minipage}}
%%%%%%%%%% End TeXmacs macros


\tmtextbf{Errors in Hopfield networks}

When a Hopfield network is trained using a list of patterns, it is desired
that those patterns are attractors, which requires that they are fixed points.
In case one of the patterns is not a fixed point, it is said that the network
has an error.

We will now derive the equations for errors in Hopfield networks.
Firstly, we shall derive the probability of error for a pattern which was used
for training, given that all the patterns were independent. Secondly, we will
derive the equation of error for a super attractor: a pattern which was used
to train the network, but which appeared in the training multiple times. The
assumption for this derivation is that the other training patterns are
independent. In the context of attachment types, we can assume that the super
attractor is the primary care giver, which was consistent in his/her
behaviour, and that the other encounters of the child were independent.

\tmtextbf{Stability of stored states, given independence of all stored
patterns}

Consider a pattern stored in the network \tmtextit{x$^k$}. We want to find out
the probability of error for that pattern: how probable is that given that
\tmtextit{x$^k$} is stored in the network it is not a fixed point?

Given the update formulae for neuron $i$ in pattern \tmtextit{k } $\left( 1
\left) \right. \right.$ and the formulae for the weights, according to the
training of the network ($2$):
\[ h_i^k = \sum^N_{j = 1} w_{i j} x_j^k  \]

\begin{equation}
  w_{\tmop{ij}} = \frac{1}{N} \sum^p_{l = 1} x^l_{i^{}} x_j^l
\end{equation}
we obtain by substitution:
\[ h_i^k = \sum^N_{j = 1} \left(^{} \frac{1}{N} \sum^p_{l = 1} x^l_{i^{}}
   x^l_j \right)_{} \cdot x_j^k = \frac{1}{N}  \sum^N_{j = 1} \sum^p_{l = 1}
   x^l_{i^{}} x^l_j x^k_{^{} j} = \]
\[ = \frac{1}{N}  \sum^N_{j = 1} x^k_{i} x^k_j x^k_{^{j}} \noplus
   \noplus + \frac{1}{N}  \sum^N_{j = 1} \sum^p_{l = 1 \nocomma, l \neq k}
   x^l_{i^{}} x^l_j x^k_{^{} j} \]
\[ = \frac{1}{N}  \sum^N_{j = 1} x^k_{i} + \frac{1}{N}  \sum^N_{j = 1}
   \sum^p_{l = 1 \nocomma, l \neq k} x^l_{i} x^l_{j} x^k_{j} = x^k_{i}
   + \frac{1}{N}  \sum^N_{j = 1} \sum^p_{l = 1 \nocomma, l \neq k} x^l_{i}
   x^l_{j} x^k_{j} \]
\[  \]
$\tmop{The} \tmop{new} \tmop{value} \tmop{of}$ neuron i $x^{k'}_i $is given by
the sign of $h_i^k $, so:




\begin{equation}
  h^k_i x^k_i \left\{ \begin{array}{l}
    \geqslant 0 \nocomma \nocomma \nocomma \tmop{if} x^k_{i'} = x^k_i\\
    < 0 \tmop{otherwise}
  \end{array} \right.^{}_{} \Rightarrow -_{} h^k_i x^k_i \left\{
  \begin{array}{l}
    \leqslant 0 \nocomma \nocomma \nocomma \tmop{if} x^k_{i'} = x^k_i\\
    > 0 \tmop{otherwise}
  \end{array} \right.
\end{equation}
\begin{equation}
  - h^k_i x^k_{^{} i} = - x^k_{i^{}} x^k_{i^{}} - x^k_{i^{}} \cdot \left(
  \frac{1}{N}  \sum^N_{j = 1} \sum^p_{l = 1 \nocomma, l \neq k} x^l_{i^{}}
  x^l_j x^k_{^{} j} \right) = 1 \begin{array}{c}
    - \underbrace{\frac{1}{N} \sum^N_{j = 1} \sum^p_{l = 1 \nocomma, l \neq k}
    x^l_{i^{}} x^l_j x^k_{^{} j} x^k_i}_{C^k_i}
  \end{array}
\end{equation}
\ \ \ \ \ \ \ \ \ \ \ \ \ \ \ \ \ \ \ \ \ \ \ \ \ \ \ \ \ \ \ \ \ \ \ \ \ \ \
\ \ \ \ \ \ \ \ \ \ \ \ \ \ \ \ \ \ \ \ \ \ \ \ \ \ \ \ \ \ \ \ \ \ \ \ \ \ \
\ \ \ \ \ \ \ \ \ \ \ \ \ \ \ \ \ \

From (3) and (4) we conclude that we get an error if $C_i^k > 1$ , so the
probability of getting an error is the given by the probability of \ $C_i^k >
1$.

We model $C_i^k $ as follows. Each $x^l_{i^{}} x^l_j x^k_{^{} j} x^k$ can be
modelled as a sample drawn from a random variable X with $\tmop{E (X) = 0}$
and $\tmop{Var} (X) = E (x^2) - E (x)^2 = 1.0$

By using the central limit theorem and by approximating $N \left( p \right.$ -
$1$) to $Np$ :
\[ \frac{1}{Np} \sum_{i = 1}^{Np} X_i \sim N \left( 0 \nocomma, \frac{1}{N p}
   \right) \]


Thus,
\[ \frac{1}{N} \sum_{i = 1}^{Np} X_i \sim N \left( 0 \nocomma, \frac{p}{N}
   \right) \]
\[ C^k_i \sim N \left( 0 \nocomma, \frac{p}{N} \right) \]
By denoting $\frac{p}{N}$ with $\sigma :$
\[ P \left( C^k_i > 1 \right) \simeq \frac{1}{\sqrt{2 \mathpi} \sigma}
   \int^{\infty}_1 e^{- \frac{x^2}{2 \sigma^2}} \tmop{dx} = \frac{1}{\sqrt{2
   \mathpi} \sigma} \int^{\infty}_{\frac{1}{\sqrt{2} \sigma}} e^{- y^2}
   \sqrt{2} \sigma \tmop{dy} = \]
\[ = \frac{1}{\sqrt{\mathpi}} \int^{\infty}_{\frac{1}{\sqrt{2} \sigma}} e^{-
   y^2} \tmop{dy} = \frac{1}{\sqrt{\mathpi}} \left( \int^{\infty}_0 e^{- y^2}
   \tmop{dy} - \int^{\frac{1}{\sqrt{2} \sigma}}_0 e^{- y^2} \tmop{dy} \right)
   = \]
\[ = \frac{1}{\sqrt{\mathpi}} \int^{\infty}_0 e^{- y^2} \tmop{dy} -
   \frac{1}{\sqrt{\mathpi}} \int^{\frac{1}{\sqrt{2} \sigma}}_0 e^{- y^2}
   \tmop{dy} = \frac{1}{\sqrt{\mathpi}} \ast \frac{\sqrt{\pi}}{2} \noplus -
   \frac{1}{2} \tmop{erf} \left( \frac{1}{\sqrt{2 \sigma}} \right) = \]
\[ = \frac{1}{2} \left( 1 \noplus \noplus - \tmop{erf} \left( \frac{1}{\sqrt{2
   \sigma}} \right) \right) = \frac{1}{2} \left( 1 \noplus \noplus -
   \tmtextit{\tmop{erf}} \left( \sqrt{\frac{N}{2 p}} \right) \right)  \]
\[ \tmop{where} \tmop{err} x = \frac{1}{\sqrt{2 \pi}} \int^x_0 e^{- x^2
   }_{^{^{}}} \tmop{dx} \]

\[  \]
\[  \]
\tmtextbf{Stability of a super attractor, given independence of all other
stored patterns}



By following the above way of reasoning, we compute the probability of error
for a super attractor.

Given that the super attractor has degree d (it appears d times in the list of
patterns used to train the network), we compute the probability that it is not
a fixed point.

Let $x^k_{}$ be the supper attractor with degree d.

Let S = $\left\{ i \left|  \right. \right. x^i_{} = x^k$ , i {\tmem{$\in
\left\{ 1 \ldots p \right\} $}}\}, be the set of indexes of occurences of the
supper attractor $x^k_{}$.

$\tmop{We} \tmop{assumed} \tmop{that} \tmop{the} \tmop{super} \tmop{attractor}
\tmop{has} \tmop{degree} d \nocomma \nocomma \nocomma \Rightarrow \left| S
\left| = d \nosymbol \right. \right.$.


\[ h_i^k = \sum^N_{j = 1} \left(^{} \frac{1}{N} \sum^p_{l = 1} x^l_{i^{}}
   x^l_j \right)_{} \cdot x_j^k = \frac{1}{N}  \sum^N_{j = 1} \sum^p_{l = 1}
   x^l_{i^{}} x^l_j x^k_{^{} j} = \]
\begin{eqnarray*}
  \frac{1}{N}  \sum^N_{j = 1} \sum_{l \in S}^p x^k_{i} x^k_j x^l_{j} \noplus \noplus + \frac{1}{N}  \sum^N_{j = 1} \sum^p_{l \nin S}
  x^l_{i} x^l_j x^k_{j}  & = &  \frac{1}{N}  \sum^N_{j = 1} \sum_{l
  \in S}^p x^k_{i} x^k_j x^k_{j} \noplus \noplus + \frac{1}{N}
  \sum^N_{j = 1} \sum^p_{l \nin S} x^l_{i} x^l_j x^k_{j} =\\
  &  & \\
  \frac{1}{N}  \sum^N_{j = 1} \sum_{l \in S}^p x^k_{i} \noplus + \noplus
  \frac{1}{N}  \sum^N_{j = 1} \sum^p_{l \nin S} x^l_{i} x^l_j x^k_{j}
  & = &  \frac{1}{N} N d x^k_{i} \noplus + \frac{1}{N}  \sum^N_{j =
  1} \sum^p_{l \nin S} x^l_{i} x^l_j x^k_{j} =
\end{eqnarray*}
\[ = d x^k_{i^{}} \noplus + \frac{1}{N}  \sum^N_{j = 1} \sum^p_{l \nin S}
   x^l_{i^{}} x^l_j x^k_{^{} j} \]
Thus
\[ - h^k_i x^k_i = - d x_i^k x_i^k - \frac{1}{N}  \sum^N_{j = 1} \sum^p_{l
   \nin S} x^l_{i^{}} x^l_j x^k_{^{} j} x^k_{i^{}} = \]
\[ - d - \underbrace{ \frac{1}{N}  \sum^N_{j = 1} \sum^p_{l \nin S} x^l_{i^{}}
   x^l_j x^k_{^{} j} x^k_{i^{}} }_{S^k_i} \]
\ \ \ \ \ \ \ \ \ \ \ \ \ \ \ \ \ \ \ \ \ \ \ \ \ \ \ \ \ \ \ \ \ \ \ \ \ \ \
\ \ \ \ \ \ \ \ \ \ \ \ \ \ \ \ \ \ \ \ \ \ \

In order to get an error, $\tmop{minus} h^k_i x^k_i \leqslant 0 \nocomma$ ,
so $S^k_i$ $\geqslant d \nosymbol$. By using the same reasoning as before,
\[ \frac{1}{N \left( p - d \right)} \sum_{i = 1}^{N \left( p - d \right)} X_i
   \sim N \left( 0 \nocomma, \frac{1}{N \left( p - d \right)} \right) \]
\[ \frac{1}{N} \sum_{i = 1}^{N \left( p - d \right)} X_i \sim N \left( 0
   \nocomma, \frac{p - d}{N} \right) \]
\[ S^k_i \sim N \left( 0 \nocomma, \frac{p - d}{N} \right) \]
By denoting $\frac{p - d}{N}$ with $\sigma :$
\[ P \left( S^k_i > d \right) \simeq \frac{1}{\sqrt{2 \mathpi} \sigma}
   \int^{\infty}_d e^{- \frac{x^2}{2 \sigma^2}} \tmop{dx} = \frac{1}{\sqrt{2
   \mathpi} \sigma} \int^{\infty}_{\frac{d}{\sqrt{2} \sigma}} e^{- y^2}
   \sqrt{2} \sigma \tmop{dy} = \]
\[ = \frac{1}{\sqrt{\mathpi}} \int^{\infty}_{\frac{d}{\sqrt{2} \sigma}} e^{-
   y^2} \tmop{dy} = \frac{1}{\sqrt{\mathpi}} \left( \int^{\infty}_0 e^{- y^2}
   \tmop{dy} - \int^{\frac{d}{\sqrt{2} \sigma}}_0 e^{- y^2} \tmop{dy} \right)
   = \]
\[ = \frac{1}{\sqrt{\mathpi}} \int^{\infty}_0 e^{- y^2} \tmop{dy} -
   \frac{1}{\sqrt{\mathpi}} \int^{\frac{d}{\sqrt{2} \sigma}}_0 e^{- y^2}
   \tmop{dy} = \frac{1}{\sqrt{\mathpi}}  \frac{\sqrt{\pi}}{2} \noplus -
   \frac{1}{2} \tmop{erf} \left( \frac{d}{\sqrt{2 \sigma}} \right) = \]
\[ = \frac{1}{2} \left( 1 \noplus \noplus - \tmop{erf} \left( \frac{1}{\sqrt{2
   \sigma}} \right) \right) = \frac{1}{2} \left( 1 \noplus \noplus -
   \tmtextit{\tmop{erf}} \left( \sqrt{\frac{N}{2 \left( p - d \right)}}
   \right) \right)  \]


\tmtextbf{Computing network parameters given maximum accepted error}

Assuming all training patterns are independent, we can use the above results
to determine the minimum number of neurons which should be used for the
network given the number of training patterns we want to store. In the case of
image recognition, this can provide to be useful as it can give a guideline
towards how to resize the images in order to bring them to the same size
(patterns which are stored in a Hopfield network need to have equal lengths).
It also enables us to compute the capacity of a network of given size, in
order to minimise errors.

Given p and N, we deduced the probability of error for the network. Thus,
conversely, we can compute the maximum ratio of p and N to ensure that the
probability of error does not exceed a maximum accepted error probability.
\[ P_{\tmop{error}} = \frac{1}{2}  \left( 1 \noplus \noplus -
   \tmtextit{\tmop{erf}} \left( \sqrt{\frac{N}{2 p}} \right) \right)  \]
\begin{eqnarray*}
  \Rightarrow \tmop{erf} \sqrt{\frac{N}{2 p}} = 1 - 2 P_{\tmop{error}} &
  \Rightarrow & \sqrt{\frac{N}{2 p}} = \tmop{inverf} \left( 1 - 2
  P_{\tmop{error}} \right)
\end{eqnarray*}
\[ \Rightarrow \frac{N}{p} = 2 \left( \tmop{inverf} \left( 1 - 2
   P_{\tmop{error}} \right) \right)^2 \]
\begin{eqnarray*}
  \frac{p}{N} = \frac{1}{2 \left( \tmop{inverf} \left( 1 - 2 P_{\tmop{error}}
  \right) \right)^2} & \tmop{and} & N = 2 p \left( \tmop{inverf} \left( 1 - 2
  P_{\tmop{error}} \right) \right)^2
\end{eqnarray*}
where inverf is the inverse of the erf function.

\tmtextbf{Measurements using the error of a network} \tmtextbf{for independent
patterns}

We will now give the reader an idea of how one should accommodate a fixed
number of patterns depending on the error accepted by the application, by
creating networks of appropriate size.

\ \ \ \ \ \ \ \ \ \ \ \ \ \ \ \ \ \ \ \ \ \ \ \ \ \ \ \ \ \

\begin{table}[h]
  \begin{tabular}{|c|c|c|}
    \hline
    \tmtextbf{p} & \tmtextbf{error} & \tmtextbf{N}\\
    \hline
    100 & 0.1 & 165\\
    \hline
    100 & 0.01 & 542\\
    \hline
    100 & 0.05 & 271\\
    \hline
    100 & 0.001 & 955\\
    \hline
  \end{tabular}
  \caption{Getting the minimum numbers of neurons required by the network by
  fixing the error and the number of patterns used to train the network.}
\end{table}

The following table describes the capacity of the network in terms of the
error accepted, by fixing the size of the network.

\begin{table}[h]
  \begin{tabular}{|c|c|c|}
    \hline
    \tmtextbf{N} & \tmtextbf{error} & \tmtextbf{p}\\
    \hline
    1000 & 0.1 & 608\\
    \hline
    1000 & 0.01 & 184\\
    \hline
    1000 & 0.05 & 359\\
    \hline
    1000 & 0.001 & 104\\
    \hline
  \end{tabular}\tmtextbf{}
  \caption{Getting the capacity of a network by fixing the error and the size
  of a network.}
\end{table}

The above results can be reproduced by using the Analysis module. For example,
by using \tmtexttt{ghci}:

{\tmkbd{\tmtexttt{}*Analysis > computeErrorSuperAttractor 10 100}}

The first argument of the function is $p$ (the number of training patterns),
and the second one is $N$ (the size of the network). A similar function can be
used for a given network \tmtexttt{computeErrorSuperAttractor}, that given a
network computes the probability of error of the super attractor. The caller
has to ensure that training patterns are independent in order for the
computation to be correct.

\tmtextbf{Decreasing the error by using one super attractor}

Above we described the derivation for obtaining the error of a super
attractor, given its degree. We remind the reader that we did this under the
assumption that all other training patterns are independent. We will now show
how the error of a pattern decreases if it is made a super attractor, by
varying the number of times it is presented to the network during training.



\tmfloat{h}{small}{table}{\begin{tabular}{|c|c|c|c|}
  \hline
  \tmtextbf{degree} & \tmtextbf{p} & \tmtextbf{N} & \tmtextbf{error}\\
  \hline
  1 & 20 & 100 & 1.08 *$10^{- 2}$\\
  \hline
  2 & 20 & 100 & 7.64*$10^{- 3}$\\
  \hline
  5 & 20 & 100 & 4.91*$10^{- 3}$\\
  \hline
  8 & 20 & 100 & $1.95 \ast 10^{- 3}$\\
  \hline
  11 & 20 & 100 & $4.29 \ast 10^{- 4}$\\
  \hline
  15 & 20 & 100 & $3.87 \ast 10^{- 6}$\\
  \hline
\end{tabular}}{} \ \ \ \ \ \ \ \ \ \ \ \ \ \ \ \ \
\tmfloat{h}{small}{table}{\begin{tabular}{|c|c|c|c|}
  \hline
  \tmtextbf{degree} & \tmtextbf{p} & \tmtextbf{N} & \tmtextbf{error}\\
  \hline
  1 & 50 & 100 & $7.65 \ast 10^{^{- 2}}$\\
  \hline
  5 & 50 & 100 & 6.80*$10^{- 2}$\\
  \hline
  10 & 50 & 100 & 5.69*10$^{- 2}$\\
  \hline
  20 & 50 & 100 & 3.39*$10^{- 2}$\\
  \hline
  30 & 50 & 100 & $1.26 \ast 10^{- 2}$\\
  \hline
  40 & 50 & 100 & $7.82 \ast 10^{- 4}$\\
  \hline
\end{tabular}}{}

Note that in the second table we are intentionally stressing the network so
that we can notice how super attractors also affect the capacity of the
network.

The above results can be reproduced by using the Analysis module. For example,
by using \tmtexttt{ghci}:

{\tmkbd{\tmtexttt{}*Analysis > computeErrorSuperAttractorNumbers 40 50 100}}

The first argument of the function is the degree of the attractor, the second
one is $p$ and the third one is $N$. A similar function can be used for a
given network \tmtexttt{computeErrorSuperAttractor}, that given a network
computes the probability of error of the super attractor. The caller of the
function has to ensure that the training patterns contain a super attractor
and that the other patterns are independent.
