
\section{Comparing Storkey and Hebbian learning}

When training a Hopfield network, we have employed two types of learning: the
first one is the Hebbian learning, while the second one is Storkey learning.
The advantage of the latter is that it increases the capacity of the network
and the basin of attraction of the clusters. We will now explain some
experiments we did with both types of learning in order to see how the
learning type affects the basin of attraction for clusters.

We will now show how the learning determines the basin of attraction of a
Gaussian distributed cluster. The generation was done using the T2 method
described in section ~\ref{sec:fexp}.

\begin{table}[h]
\centering
  \begin{tabular}{|c|c|c|c|c|c|}
    \hline
    \tmtextbf{Learning} & N & Cluster size & $\mu$ & $\sigma$ & Average size
    of basin of attraction\\
    \hline
    Hebbian & 50 & 2 & 25 & 5 & 10.0\\
    \hline
    Storkey & 50 & 2 & 25 & 5 & 18.0\\
    \hline
    Hebbian & 50 & 4 & 25 & 5 & 1.5\\
    \hline
    Storkey & 50 & 4 & 25 & 5 & 4\\
    \hline
    Hebbian & 50 & 4 & 25 & 10 & 5.75\\
    \hline
    Storkey & 50 & 4 & 25 & 10 & 8.0\\
    \hline
    Hebbian & 50 & 5 & 25 & 5 & 4.4\\
    \hline
    Storkey & 50 & 5 & 25 & 5 & 4.8\\
    \hline
    Hebbian & 50 & 6 & 25 & 5 & 4.2\\
    \hline
    Storkey & 50 & 6 & 25 & 5 & 4.4\\
    \hline
    Hebbian & 50 & 6 & 25 & 10 & 4.45\\
    \hline
    Storkey & 50 & 6 & 25 & 10 & 5.61\\
    \hline
  \end{tabular}
  \caption{Results comparing Storkey and Hebbian learning for various parameters}
\end{table}



The results confirm what the mathematical theory showed us: Storkey learning
increases the average basin of attraction. We must note that this does not
come without a price: training the network using Storkey learning slowed down
our experiments, as it is more expensive. Depending to the application, this
is an acceptable trade off. All our functions and experiments can be performed
with both types of learning, just by changing a parameter (the learning type),
which enables any user of our libraries to make the choice depending on the
use case.

%T1

\begin{figure}[h]
  \centering
  \begin{tikzpicture}
\begin{axis}[
xmin = 0,
ymin = 0,
xlabel = $p$-value,
ylabel = Average Basin Size
]

\addplot [
color = blue,
mark = *, % A filled circle
only marks,
smooth  % draw smooth curve
] table [
y = mean
] {data-T1-onecluster-100.csv};

\addlegendentry{Hebbian Learning} ;


\addplot [
color = red,
mark = *, % A filled circle
only marks,
smooth  % draw smooth curve
] table [
y = mean
] {data-storkey-T1-onecluster-100.csv};

\addlegendentry{Storkey Learning} ;

\end{axis}
\end{tikzpicture}

\caption{Comparing the impact of different learning methods on the average basin size of patterns belonging to a cluster generated using T1.}
\label{fig:plot-storkey-T1-onecluster}
\end{figure}

\begin{figure}[h]
  \centering
  \begin{tikzpicture}
\begin{axis}[
xmin = 0,
xmax = 0.5,
ymin = 0,
xlabel = p-value of cluster 2,
ylabel = Average Basin Size of Cluster 2,
width=12cm,
height=9cm
]

\addlegendentry{Cluster 1 with $p$ = 0.2} ;

\addplot [
color = red,
mark = x, % A filled circle
% mark = none,
only marks,
smooth  % draw smooth curve
] table [
y = mean-new
] {data-storkey-T1-twocluster-100-0.4.csv};

\addlegendentry{Cluster 1 with $p$ = 0.4} ;


\addplot [
color = blue,
mark = triangle, % A filled circle
only marks,
smooth  % draw smooth curve
] table [
y = mean-new
] {data-storkey-T1-twocluster-100-0.2.csv};

\addlegendentry{Cluster 1 with $p$ = 0.2} ;



\end{axis}
\end{tikzpicture}

\caption{Comparing the impact of different learning methods on the average basin size of patterns belonging to two clusters generated using T1. The first cluster has fixed $p = 0.2$}
\label{fig:plot-storkey-T1-twoclusters}
\end{figure}


%T2

\begin{figure}[h]
  \centering
  \begin{tikzpicture}
\begin{axis}[
xmin = 0,
ymin = 0,
xlabel = Standard deviation,
ylabel = Average Basin Size
]

\addplot [
color = blue,
mark = *, % A filled circle
only marks,
smooth  % draw smooth curve
] table [
y = mean
] {data-T2-onecluster-100.csv};

\addlegendentry{Hebbian Learning} ;

\addplot [
color = red,
mark = *, % A filled circle
only marks,
smooth  % draw smooth curve
] table [
y = mean
] {data-storkey-T2-onecluster-100.csv};

\addlegendentry{Storkey Learning} ;


\end{axis}
\end{tikzpicture}

\caption{Comparing the impact of different learning methods on the average basin size of patterns belonging to a cluster generated using T2.}
\label{fig:plot-storkey-T2-onecluster}
\end{figure}

\begin{figure}[h]
  \centering
  \begin{tikzpicture}
\begin{axis}[
xmin = 0,
ymin = 0,
xlabel = Standard Deviation of Cluster 2,
ylabel = Average Basin Size of Cluster 2,
width=12cm,
height=9cm
]

\addplot [
color = red,
mark = x, % A filled circle
% mark = none,
only marks,
smooth  % draw smooth curve
] table [
y = mean-new
] {data-storkey-T2-twocluster-100-5.csv};

\addlegendentry{Cluster 1 with $\sigma$ = 5.0} ;


\addplot [
color = blue,
mark = triangle, % A filled circle
only marks,
smooth  % draw smooth curve
] table [
y = mean-new
] {data-storkey-T2-twocluster-100-10.csv};

\addlegendentry{Cluster 1 with $\sigma$ = 10.0} ;



\end{axis}
\end{tikzpicture}

\caption{Comparing the impact of different learning methods on the average basin size of patterns belonging to two clusters generated using T2. The first cluster has fixed $\sigma = 10$}
\label{fig:plot-storkey-T2-twoclusters}
\end{figure}
