% IMPORTANT: Please write various parts in different files, and then include
% them into this document.
% If you have a file called intro.tex then write: \include{intro}
% This is to avoid nasty merge conflicts, as well as to keep it tidy,
% modular, etc

\documentclass[11pt,a4paper,oneside]{report}


% Make bibliography appear in table of contents
\usepackage[nottoc,numbib]{tocbibind}



\title{Attractor Neural Networks for Modelling Associative Memory}
\date{January 2012}
\author{
  Wael Aljieshi\\
  \texttt{first2.last2@xxxxx.com}
  \and
  Niklas Hambuechen\\
  \texttt{first2.last2@xxxxx.com}
  \and
  Razvan Marinesuc\\
  \texttt{rvm10@imperial.ac.uk}
  \and
  Mihaela Rosca\\
  \texttt{first2.last2@xxxxx.com}
  \and
  Lukasz Severyn\\
  \texttt{first1.last1@xxxxx.com}
}


\begin{document}
% General notes:

% Report Guidlines:
% http://www.doc.ic.ac.uk/lab/thirdyear/group-project/ReportGuidelines2012.pdf

% "This is more marketing than engineering. We don't want a diary"

% Tony: "How long? I don't know. Keep it short. 30 pages is absolutely fine.
% 40 a bit too much. 20 is a bit..."
% NOTE this excludes the appendices

% Keep it simple and to the point. Don't waffle

% Report is read by people who DONT know your work, though they are
% technically minded


\maketitle{}


\renewcommand{\abstractname}{Executive Summary}
\begin{abstract}
% Compulsory
% 1 page max. Tony: "third to half a page ideal probably"
% Anandha: "Don't put a photo of your group in the executive summary or
% something like that!"
% Wael: I would say it might be appropriate to do something like that in the
% presentation though. We can discuss and see.

Attractor Neural networks have been a subject of intensive research in the past 30 years. They have been proposed as a model of associative memory and have been used in applications such as facial and speech recognition or for modelling biological activities or the human brain. 


The task of our project was to get a deeper understanding of these networks by simulating them and inverstigating their properties. For this purpose, we have showed their importance by implementing an image recognition software. Furthermore, we have also used them for modelling a psychological concept called Attachment Theory, in order to better understand the human mind and how to cure various mental disorders. 

Our project is trying to model, on a methaforical level, how an individual will react given various positive or negative memories. We have implemented two mathematical models of neural networks, that can simulate various functions of the brain. Amongst these, two of them are of uttermost importance: learning and recalling information. We analysed how easy different memories are recalled, how probable it is for one memory to be recalled, or how some memories can be forgotten or reinforced. The latter is extemely important for the psychological studies, since it provides a way of curing some mental disorders that have been caused by negative memories. 

% should we include some of the results as well? 
\end{abstract}


\tableofcontents

\chapter{Introduction}
% Tony: "1, 2, 3 pages"
% What is the problem?
% Why is it interesting?
% How did you solve it?
% How far did you get?

% Includes motivation, objectives, and contributions (what you achieved)
\section{Motivation}
The human brain is one of the most complex objects, that researchers have been trying to analyse since antiquity. Understanding how it works, by performing experiments and modelling it would help us cure various brain and mental disorders. 

Several medical discoveries from the 20th century have enlightened our knowledge of the human brain, and therefore different mathematical models of artificial neural networks have been created, inspired from biology and medicine. 

Our motivation is to use some of these neural networks to explore the mathematics of a developing theory called Self-Attachment. This theory is aiming to help curing various mental problems that people are currently facing.Recent research has shown that a mathematical way of analysing these disorders is starting to become feasible. \cite{net_model_neuroses}

The subsequent chapters will introduce the Attachment Theory and the mathematical model. We have mainly focused on the technical side, and described  psychological analogies that explain our technical results. 

\section{Objectives}

Our main objective is to confirm the results of previous work done by Federico Macinelli, who has been analysing the attachment theory using neural networks. He has performed several experiments regarding clusters of attractors, basin sizes, and gaussian-distributed patterns. Furtheremore, we have been aiming at extending his results by exploring the following concepts:
\begin{itemize}
\item Using the network for performing image recognition
\item Restricted Bolzman machines
\item Super-attractors
\end{itemize}


In addition to that, we were aiming to improve some of the methods that were used in his experiments. This includes the technique for sampling gaussian-distributed patterns or for calculating basin sizes. Our final aims consisted of explaining some of the subsequeny inconsistencies that have been found in his results. 

\section{Contributions}

Since our project was related to exploring attractor-based neural networks, we have obtained interesting results about their attractors. These attractors are analogous to learned memories or experiences that an individual has learned. 


Our main contributions are outlined below:
\begin{itemize}
\item Confirming Federico's results by reimplementing them in a completely different evironment (Haskell)
\item Proving that the Hopfield network is capable of performing image recognition, by learning image patterns and then recalling the closest image, when queried for an input. Furthermore, we have proven it's associative memory properties, by recalling some of the learned images. 
\item We found out that training patterns are not guaranteed to become fixed points in the Hopfield network. This is an aspect that was not mentioned in simmilar research papers, and we first thought the the patterns are always fixed points.
\item Extending some of the experiments to the Bolzman Machine, which provides a nice way of overcoming some limitations of the Hopfield Network. It can prevent convergence to spurious patterns by using a stochasting update rule. 
\item A thorough analysis of Super-Attractors, which represent patterns that have been used multiple times in the training process. They generally have greater basin sizes, and therefore patterns will have a greater chance of converging to them compared to normal attractors. 
\end{itemize}

\chapter{Design and Implementation}
% Justify design decisions, tools used, etc.
% NO CODE!!! (except for skeleton of algorithms maybe)

\section{Background Research}
This chapter details the background research we have done in order to understand the Hopfield Neural Networks, realise their importance in various applications and to inverstigate the feasability of our project. 

\subsection{Attachment Theory}

\subsection{Using Neural networks to diagnose mental disorders}



\subsection{Hopfield Networks}

\subsubsection{Correlated Patterns}

\subsubsection{Super-Attractors}


\chapter{Evaluation}
% Step back and say things like:
% "X worked out well"
% "in hindsight we could have used this tool instead"

% How do we know that our code works?
% e.g. functionality + performance testing
% show relevant testing *results*

% In a nutshell, how successful was this project
% Tony: "DONT SAY: all our objectives (from intro) were met. Have the maturity
% to show your mistakes and more importantly what you learned!"


\chapter{Conclusion and Future Extensions}
% Tony: "Conclusion is NOT a summary"
% e.g. result that you did not know about

\chapter{Project Management}
% Use this to convey your *group collaboration*
% Team organisation, supervisor, software engineering team skills

% DONT give lots of text about the SE process (a la diary), rather your team,
% deliverables, etc

% Note that group collaboration is worth 30%. This is part assessed by our
% supervisor comments, how we present ourselves, and probably a bit will be
% influenced by what we say here. Bare this in mind!


\nocite{*} % Show all Bib-entries
\bibliographystyle{plain}
\bibliography{main}



\chapter{Appendices}
% these do NOT count as part of the suggested page count
% This is probably a good place to explain the models in some detail for example


\end{document}
