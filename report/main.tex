% IMPORTANT: Please write various parts in different files, and then include
% them into this document.
% If you have a file called intro.tex then write: \include{intro}
% This is to avoid nasty merge conflicts, as well as to keep it tidy,
% modular, etc

\documentclass[11pt,a4paper,oneside]{report}


% Make bibliography appear in table of contents
\usepackage[nottoc,numbib]{tocbibind}



\title{Attractor Neural Networks for Modelling Associative Memory}
\date{January 2012}
\author{
  Name 1\\
  \texttt{first1.last1@xxxxx.com}
  \and
  Name2\\
  \texttt{first2.last2@xxxxx.com}
}


\begin{document}
% General notes:

% Report Guidlines:
% http://www.doc.ic.ac.uk/lab/thirdyear/group-project/ReportGuidelines2012.pdf

% "This is more marketing than engineering. We don't want a diary"

% Tony: "How long? I don't know. Keep it short. 30 pages is absolutely fine.
% 40 a bit too much. 20 is a bit..."
% NOTE this excludes the appendices

% Keep it simple and to the point. Don't waffle

% Report is read by people who DONT know your work, though they are
% technically minded


\maketitle{}


\renewcommand{\abstractname}{Executive Summary}
\begin{abstract}
% Compulsory
% 1 page max. Tony: "third to half a page ideal probably"
% Anandha: "Don't put a photo of your group in the executive summary or
% something like that!"
% Wael: I would say it might be appropriate to do something like that in the
% presentation though. We can discuss and see.
\end{abstract}


\tableofcontents

\chapter{Introduction}
% Tony: "1, 2, 3 pages"
% What is the problem?
% Why is it interesting?
% How did you solve it?
% How far did you get?

% Includes motivation, objectives, and contributions (what you achieved)


\chapter{Design and Implementation}
% Justify design decisions, tools used, etc.
% NO CODE!!! (except for skeleton of algorithms maybe)



\chapter{Evaluation}
% Step back and say things like:
% "X worked out well"
% "in hindsight we could have used this tool instead"

% How do we know that our code works?
% e.g. functionality + performance testing
% show relevant testing *results*

% In a nutshell, how successful was this project
% Tony: "DONT SAY: all our objectives (from intro) were met. Have the maturity
% to show your mistakes and more importantly what you learned!"


\chapter{Conclusion and Future Extensions}
% Tony: "Conclusion is NOT a summary"
% e.g. result that you did not know about

\chapter{Project Management}
% Use this to convey your *group collaboration*
% Team organisation, supervisor, software engineering team skills

% DONT give lots of text about the SE process (a la diary), rather your team,
% deliverables, etc

% Note that group collaboration is worth 30%. This is part assessed by our
% supervisor comments, how we present ourselves, and probably a bit will be
% influenced by what we say here. Bare this in mind!


\nocite{*} % Show all Bib-entries
\bibliographystyle{plain}
\bibliography{main}



\chapter{Appendices}
% these do NOT count as part of the suggested page count
% This is probably a good place to explain the models in some detail for example


\end{document}
